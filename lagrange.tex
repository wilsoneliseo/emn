% pagina 114 seccion 3.1 --> 1b, d, 5b,d, 6b, 9, 10, 12, 13b,d, 18, 19
\section{ejercicio 1}
Para las siguientes funciones $f(x)$, sea $x_0=0$, $x_1=0.6$ y
$x_2=0.9$. Construir el polinomio de interpolación, de grado adecuado,
para aproximar $f(0.45)$, y encuentre el error absoluto.


\subsection{inciso b)}
$f(x)=\sqrt{1+x}$

\begin{verbatim}
  f(x):=sqrt(1+x);
  a:matrix([0,0.6]);
  a:addrow(a,map(f,a));
  polagrange(a);
\end{verbatim}
la tabla es 
$$\pmatrix{0&0.6\cr 1.0&1.264911064067352\cr }$$
haciendo los calculos

$${{\mbox{{}(1.0){}}\,\left(x-\mbox{{}0.6{}}\right)}\over{
 \mbox{{}0{}}-\mbox{{}0.6{}}}}+{{\mbox{{}(1.264911064067352){}}\,
 \left(x-\mbox{{}0{}}\right)}\over{\mbox{{}0.6{}}-\mbox{{}0{}}}}={{
 8268424\,x+18727245}\over{18727245}}$$

$$p\left(x\right)={{8268424\,x+18727245}\over{18727245}}$$
$$p\left(0.45\right)=1.198683298050514$$
$$f\left(0.45\right)=1.20415945787923$$

$$
error\;absoluto=|f(0.45)-p(0.45)|=0.0054761598287154
$$

\subsection{inciso d)}
$f(x)=\tan(x)$

\begin{verbatim}
  f(x):=tan(x);
  a:matrix([0,0.6]);
  a:addrow(a,map(f,a));
  polagrange(a);
\end{verbatim}
la tabla es 
$$\pmatrix{0&0.6\cr 0&0.68413680834169\cr }$$
haciendo los calculos
$${{\mbox{{}(0){}}\,\left(x-\mbox{{}0.6{}}\right)}\over{\mbox{{}0{}}-
 \mbox{{}0.6{}}}}+{{\mbox{{}(0.68413680834169){}}\,\left(x-
 \mbox{{}0{}}\right)}\over{\mbox{{}0.6{}}-\mbox{{}0{}}}}={{8078723\,x
 }\over{7085182}}$$
$$p\left(x\right)={{8078723\,x}\over{7085182}}$$
$$p\left(0.45\right)=0.51310260625627$$
$$f\left(0.45\right)=0.48305506561658$$
$$error\;absoluto=|f\left(0.45\right)-p\left(0.45\right)| =0.03004754063969$$

\section{ejercicio 5}
Use polinomios de interpolación de Lagrange de grados uno, dos y tres
para aproximar lo siguiente:

\subsection{inciso b)}
$f\left(-1/3\right)$ con los datos
$$\pmatrix{-0.75&-0.5&-0.25&0\cr -0.0718125&-0.02475&0.3349375&1.101
 \cr }$$

\subsubsection{Grado uno:}
\begin{verbatim}
  a:matrix([-0.5,-0.25],[-0.02475,0.3349375]);
  polagrange(a);
\end{verbatim}
los datos a utilizar
$$\pmatrix{-0.5&-0.25\cr -0.02475&0.3349375\cr }$$
haciendo los calculos
$${{\mbox{{}(0.3349375){}}\,\left(x-\mbox{{}-0.5{}}\right)}\over{
 \mbox{{}-0.25{}}-\mbox{{}-0.5{}}}}+{{\mbox{{}(-0.02475){}}\,\left(x-
 \mbox{{}-0.25{}}\right)}\over{\mbox{{}-0.5{}}-\mbox{{}-0.25{}}}}={{
 11510\,x+5557}\over{8000}}$$
$$p\left(x\right)={{11510\,x+5557}\over{8000}}$$

$$p\left(-{{1}\over{3}}\right)={{5161}\over{24000}}=0.21504166666667$$

\subsubsection{Grado dos:}
\begin{verbatim}
  a:matrix([-0.5,-0.25,0],[-0.02475,0.3349375,1.101]);
  polagrange(a);
\end{verbatim}
los datos a utilizar
$$\pmatrix{-0.5&-0.25&0\cr -0.02475&0.3349375&1.101\cr }$$
haciendo los calculos

\begin{eqnarray}
{{\mbox{{}0.3349375{}}\,\left(x-\mbox{{}-0.5{}}\right)\,\left(x-
 \mbox{{}0{}}\right)}\over{\left(\mbox{{}-0.25{}}-\mbox{{}-0.5{}}
 \right)\,\left(\mbox{{}-0.25{}}-\mbox{{}0{}}\right)}}+{{
 \mbox{{}-0.02475{}}\,\left(x-\mbox{{}-0.25{}}\right)\,\left(x-
 \mbox{{}0{}}\right)}\over{\left(\mbox{{}-0.5{}}-\mbox{{}-0.25{}}
 \right)\,\left(\mbox{{}-0.5{}}-\mbox{{}0{}}\right)}}+ \nonumber \\{{
 \mbox{{}1.101{}}\,\left(x-\mbox{{}-0.25{}}\right)\,\left(x-
 \mbox{{}-0.5{}}\right)}\over{\left(\mbox{{}0{}}-\mbox{{}-0.25{}}
 \right)\,\left(\mbox{{}0{}}-\mbox{{}-0.5{}}\right)}}={{3251\,x^2+
 3877\,x+1101}\over{1000}\nonumber}
\end{eqnarray}
$$p\left(x\right)={{3251\,x^2+3877\,x+1101}\over{1000}}$$
$$p\left(-{{1}\over{3}}\right)={{1529}\over{9000}}=0.16988888888889$$

\subsubsection{Grado tres:}
\begin{verbatim}
  a:matrix([-0.75,-0.5,-0.25,0],[-0.0718125,-0.02475,0.3349375,1.101]);
  polagrange(a);
\end{verbatim}
se utiliza todos los datos de la tabla original
$$\pmatrix{-0.75&-0.5&-0.25&0\cr -0.0718125&-0.02475&0.3349375&1.101
 \cr }$$
haciendo los calculos

\begin{eqnarray}
{{\mbox{{}0.3349375{}}\,\left(x-\mbox{{}-0.5{}}\right)\,\left(x-
 \mbox{{}-0.75{}}\right)\,\left(x-\mbox{{}0{}}\right)}\over{\left(
 \mbox{{}-0.25{}}-\mbox{{}-0.5{}}\right)\,\left(\mbox{{}-0.25{}}-
 \mbox{{}-0.75{}}\right)\,\left(\mbox{{}-0.25{}}-\mbox{{}0{}}\right)
 }}\nonumber \\+{{\mbox{{}-0.02475{}}\,\left(x-\mbox{{}-0.25{}}\right)\,\left(x-
 \mbox{{}-0.75{}}\right)\,\left(x-\mbox{{}0{}}\right)}\over{\left(
 \mbox{{}-0.5{}}-\mbox{{}-0.25{}}\right)\,\left(\mbox{{}-0.5{}}-
 \mbox{{}-0.75{}}\right)\,\left(\mbox{{}-0.5{}}-\mbox{{}0{}}\right)}}
 \nonumber \\+{{\mbox{{}-0.0718125{}}\,\left(x-\mbox{{}-0.25{}}\right)\,\left(x-
 \mbox{{}-0.5{}}\right)\,\left(x-\mbox{{}0{}}\right)}\over{\left(
 \mbox{{}-0.75{}}-\mbox{{}-0.25{}}\right)\,\left(\mbox{{}-0.75{}}-
 \mbox{{}-0.5{}}\right)\,\left(\mbox{{}-0.75{}}-\mbox{{}0{}}\right)}}
 \nonumber \\+{{\mbox{{}1.101{}}\,\left(x-\mbox{{}-0.25{}}\right)\,\left(x-
 \mbox{{}-0.5{}}\right)\,\left(x-\mbox{{}-0.75{}}\right)}\over{\left(
 \mbox{{}0{}}-\mbox{{}-0.25{}}\right)\,\left(\mbox{{}0{}}-
 \mbox{{}-0.5{}}\right)\,\left(\mbox{{}0{}}-\mbox{{}-0.75{}}\right)}} \nonumber \\=
 {{1000\,x^3+4001\,x^2+4002\,x+1101}\over{1000}\nonumber}
\end{eqnarray}

$$p\left(x\right)={{1000\,x^3+4001\,x^2+4002\,x+1101}\over{1000}}$$

$$p\left(-{{1}\over{3}}\right)={{589}\over{3375}}=0.17451851851852$$

\subsection{inciso d)}
$f\left(0.9\right)$ con los datos
$$\pmatrix{0.6&0.7&0.8&1.0\cr -0.1769446&0.01375227&0.22363362&
 0.65809197\cr }$$

\subsubsection{Grado uno:}
\begin{verbatim}
  a:matrix([0.8,1.0],[0.22363362,0.65809197]);
  polagrange(a);
\end{verbatim}
los datos a utilizar
$$\pmatrix{0.8&1.0\cr 0.22363362&0.65809197\cr }$$
haciendo los calculos
$${{\mbox{{}(0.22363362){}}\,\left(x-\mbox{{}1.0{}}\right)}\over{
\mbox{{}0.8{}}-\mbox{{}1.0{}}}}+{{\mbox{{}(0.65809197){}}\,\left(x-
\mbox{{}0.8{}}\right)}\over{\mbox{{}1.0{}}-\mbox{{}0.8{}}}}={{
368539439989973\,x-256891063989973}\over{169654670000000}}$$

$$p\left(x\right)={{368539439989973\,x-256891063989973}\over{
169654670000000}}$$

$$p\left(0.9\right)=0.440862795$$

\subsubsection{Grado dos:}
\begin{verbatim}
  a:matrix([0.7,0.8,1.0],[0.0138,0.224,0.658]);
  polagrange(a);
\end{verbatim}
los datos a utilizar
$$\pmatrix{0.7&0.8&1.0\cr 0.0138&0.224&0.658\cr }$$
haciendo los calculos

\begin{eqnarray}
{{\mbox{{}0.01375227{}}\,\left(x-\mbox{{}0.8{}}\right)\,\left(x-
 \mbox{{}1.0{}}\right)}\over{\left(\mbox{{}0.7{}}-\mbox{{}0.8{}}
 \right)\,\left(\mbox{{}0.7{}}-\mbox{{}1.0{}}\right)}}+{{
 \mbox{{}0.22363362{}}\,\left(x-\mbox{{}0.7{}}\right)\,\left(x-
 \mbox{{}1.0{}}\right)}\over{\left(\mbox{{}0.8{}}-\mbox{{}0.7{}}
 \right)\,\left(\mbox{{}0.8{}}-\mbox{{}1.0{}}\right)}}\nonumber\\+{{
 \mbox{{}0.65809197{}}\,\left(x-\mbox{{}0.7{}}\right)\,\left(x-
 \mbox{{}0.8{}}\right)}\over{\left(\mbox{{}1.0{}}-\mbox{{}0.7{}}
 \right)\,\left(\mbox{{}1.0{}}-\mbox{{}0.8{}}\right)}}={{24492750\,x^
 2+173142225\,x-131825778}\over{100000000}\nonumber}
\end{eqnarray}

$$p\left(x\right)={{24492750\,x^2+173142225\,x-131825778}\over{
 100000000}}$$

$$p\left(0.9\right)=0.43841352$$

\subsubsection{Grado tres:}
\begin{verbatim}
  a:matrix([0.6,0.7,0.8,1.0],[-0.1769446,0.01375227,0.22363362,0.65809197]);
  polagrange(a);
\end{verbatim}
se utiliza los datos originales
$$\pmatrix{0.6&0.7&0.8&1.0\cr -0.1769446&0.01375227&0.22363362& 0.65809197\cr }$$
haciendo los calculos
\begin{eqnarray}
  {{\mbox{{}-0.1769446{}}\,\left(x-\mbox{{}0.7{}}\right)\,\left(x-
 \mbox{{}0.8{}}\right)\,\left(x-\mbox{{}1.0{}}\right)}\over{\left(
 \mbox{{}0.6{}}-\mbox{{}0.7{}}\right)\,\left(\mbox{{}0.6{}}-
 \mbox{{}0.8{}}\right)\,\left(\mbox{{}0.6{}}-\mbox{{}1.0{}}\right)}}\nonumber\\+
 {{\mbox{{}0.01375227{}}\,\left(x-\mbox{{}0.6{}}\right)\,\left(x-
 \mbox{{}0.8{}}\right)\,\left(x-\mbox{{}1.0{}}\right)}\over{\left(
 \mbox{{}0.7{}}-\mbox{{}0.6{}}\right)\,\left(\mbox{{}0.7{}}-
 \mbox{{}0.8{}}\right)\,\left(\mbox{{}0.7{}}-\mbox{{}1.0{}}\right)}}\nonumber\\+
 {{\mbox{{}0.22363362{}}\,\left(x-\mbox{{}0.6{}}\right)\,\left(x-
 \mbox{{}0.7{}}\right)\,\left(x-\mbox{{}1.0{}}\right)}\over{\left(
 \mbox{{}0.8{}}-\mbox{{}0.6{}}\right)\,\left(\mbox{{}0.8{}}-
 \mbox{{}0.7{}}\right)\,\left(\mbox{{}0.8{}}-\mbox{{}1.0{}}\right)}}\nonumber\\+
 {{\mbox{{}0.65809197{}}\,\left(x-\mbox{{}0.6{}}\right)\,\left(x-
 \mbox{{}0.7{}}\right)\,\left(x-\mbox{{}0.8{}}\right)}\over{\left(
 \mbox{{}1.0{}}-\mbox{{}0.6{}}\right)\,\left(\mbox{{}1.0{}}-
 \mbox{{}0.7{}}\right)\,\left(\mbox{{}1.0{}}-\mbox{{}0.8{}}\right)}}\nonumber\\=
 -{{357148250\,x^3-941856125\,x^2+389440945\,x+63648536}\over{
 200000000}\nonumber}
\end{eqnarray}

$$p\left(x\right)=-{{357148250\,x^3-941856125\,x^2+389440945\,x+
 63648536}\over{200000000}}$$

$$p\left(0.9\right)=0.4419850025$$

\section{ejercicio 6}

\subsection{inciso b)}
Use la aproximación por polinomio de Lagrange de grados uno, dos y tres para
$f(0)$ utilizando los datos:
$$\pmatrix{-0.5&-0.25&0.25&0.5\cr 1.9375&1.33203&0.800781&0.6875\cr }$$

\subsubsection{Grado 1:}
\begin{verbatim}
  a:matrix([-0.25,0.25],[1.33203,0.800781]);
  polagrange(a);
\end{verbatim}
se toman dos pares de puntos adecuados para la aproximación
$$\pmatrix{-0.25&0.25\cr 1.33203&0.800781\cr }$$
haciendo los calculos
$${{\mbox{{}(1.33203){}}\,\left(x-\mbox{{}0.25{}}\right)}\over{
 \mbox{{}-0.25{}}-\mbox{{}0.25{}}}}+{{\mbox{{}(0.800781){}}\,\left(x-
 \mbox{{}-0.25{}}\right)}\over{\mbox{{}0.25{}}-\mbox{{}-0.25{}}}}=-{{
 2124996\,x-2132811}\over{2000000}}$$

$$p\left(x\right)=-{{2124996\,x-2132811}\over{2000000}}$$
$$p\left(0\right)={{2132811}\over{2000000}}=1.0664055$$

\subsubsection{Grado dos:}
\begin{verbatim}
  a:matrix([-0.25,0.25,0.5],[1.33203,0.800781,0.6875]);
  polagrange(a);
\end{verbatim}
se escojen los puntos mas adecuados
$$\pmatrix{-0.25&0.25&0.5\cr 1.33203&0.800781&0.6875\cr }$$
haciendo los calculos

\begin{eqnarray}
  {{\mbox{{}1.33203{}}\,\left(x-\mbox{{}0.25{}}\right)\,\left(x-
 \mbox{{}0.5{}}\right)}\over{\left(\mbox{{}-0.25{}}-\mbox{{}0.25{}}
 \right)\,\left(\mbox{{}-0.25{}}-\mbox{{}0.5{}}\right)}}+{{
 \mbox{{}0.800781{}}\,\left(x-\mbox{{}-0.25{}}\right)\,\left(x-
 \mbox{{}0.5{}}\right)}\over{\left(\mbox{{}0.25{}}-\mbox{{}-0.25{}}
 \right)\,\left(\mbox{{}0.25{}}-\mbox{{}0.5{}}\right)}}\nonumber\\+{{
 \mbox{{}0.6875{}}\,\left(x-\mbox{{}-0.25{}}\right)\,\left(x-
 \mbox{{}0.25{}}\right)}\over{\left(\mbox{{}0.5{}}-\mbox{{}-0.25{}}
 \right)\,\left(\mbox{{}0.5{}}-\mbox{{}0.25{}}\right)}}={{2437496\,x^
 2-3187494\,x+3046873}\over{3000000}\nonumber}
\end{eqnarray}

$$p\left(x\right)={{2437496\,x^2-3187494\,x+3046873}\over{3000000}}$$

$$p\left(0\right)={{3046873}\over{3000000}}=1.015624333333334$$

\subsubsection{Grado tres:}
\begin{verbatim}
  a:matrix([-0.5,-0.25,0.25,0.5],[1.9375,1.33203,0.800781,0.6875]);
  polagrange(a);
\end{verbatim}
se utiliza las cuatro pares de puntos originales
$$\pmatrix{-0.5&-0.25&0.25&0.5\cr 1.9375&1.33203&0.800781&0.6875\cr }$$
haciendo los calculos
\begin{eqnarray}
  {{\mbox{{}1.33203{}}\,\left(x-\mbox{{}-0.5{}}\right)\,\left(x-
 \mbox{{}0.25{}}\right)\,\left(x-\mbox{{}0.5{}}\right)}\over{\left(
 \mbox{{}-0.25{}}-\mbox{{}-0.5{}}\right)\,\left(\mbox{{}-0.25{}}-
 \mbox{{}0.25{}}\right)\,\left(\mbox{{}-0.25{}}-\mbox{{}0.5{}}\right)
 }}\nonumber\\+{{\mbox{{}1.9375{}}\,\left(x-\mbox{{}-0.25{}}\right)\,\left(x-
 \mbox{{}0.25{}}\right)\,\left(x-\mbox{{}0.5{}}\right)}\over{\left(
 \mbox{{}-0.5{}}-\mbox{{}-0.25{}}\right)\,\left(\mbox{{}-0.5{}}-
 \mbox{{}0.25{}}\right)\,\left(\mbox{{}-0.5{}}-\mbox{{}0.5{}}\right)
 }}\nonumber\\+{{\mbox{{}0.800781{}}\,\left(x-\mbox{{}-0.25{}}\right)\,\left(x-
 \mbox{{}-0.5{}}\right)\,\left(x-\mbox{{}0.5{}}\right)}\over{\left(
 \mbox{{}0.25{}}-\mbox{{}-0.25{}}\right)\,\left(\mbox{{}0.25{}}-
 \mbox{{}-0.5{}}\right)\,\left(\mbox{{}0.25{}}-\mbox{{}0.5{}}\right)
 }}\nonumber\\+{{\mbox{{}0.6875{}}\,\left(x-\mbox{{}-0.25{}}\right)\,\left(x-
 \mbox{{}-0.5{}}\right)\,\left(x-\mbox{{}0.25{}}\right)}\over{\left(
 \mbox{{}0.5{}}-\mbox{{}-0.25{}}\right)\,\left(\mbox{{}0.5{}}-
 \mbox{{}-0.5{}}\right)\,\left(\mbox{{}0.5{}}-\mbox{{}0.25{}}\right)
 }}\nonumber\\=-{{1500016\,x^3-1968756\,x^2+1499996\,x-1476561}\over{1500000}\nonumber}
\end{eqnarray}

$$p\left(x\right)=-{{1500016\,x^3-1968756\,x^2+1499996\,x-1476561}\over{1500000}}$$

$$p\left(0\right)={{492187}\over{500000}}=0.984374$$

\section{ejercicio 9}
Sea $P_3(x)$ el polinomio de interpolación de los datos $(0,\,0)$,
$(0.5,\;y)$, $(1,\;3)$ y $(2,\;2)$. El coeficiente de $x^3$ en
$P_3(x)$ es $6$. Encuentre $y$.\\

Haciendo las operaciones correspondientes a un polinomio de grado tres

\begin{eqnarray}
  {{\left(x-\mbox{{}0{}}\right)\,\left(x-\mbox{{}1{}}\right)\,\left(x
 -\mbox{{}2{}}\right)\,y}\over{\left(\mbox{{}0.5{}}-\mbox{{}0{}}
 \right)\,\left(\mbox{{}0.5{}}-\mbox{{}1{}}\right)\,\left(
 \mbox{{}0.5{}}-\mbox{{}2{}}\right)}}\nonumber\\+{{\mbox{{}0{}}\,\left(x-
 \mbox{{}0.5{}}\right)\,\left(x-\mbox{{}1{}}\right)\,\left(x-
 \mbox{{}2{}}\right)}\over{\left(\mbox{{}0{}}-\mbox{{}0.5{}}\right)\,
 \left(\mbox{{}0{}}-\mbox{{}1{}}\right)\,\left(\mbox{{}0{}}-
 \mbox{{}2{}}\right)}}\nonumber\\+{{\mbox{{}3{}}\,\left(x-\mbox{{}0{}}\right)\,
 \left(x-\mbox{{}0.5{}}\right)\,\left(x-\mbox{{}2{}}\right)}\over{
 \left(\mbox{{}1{}}-\mbox{{}0{}}\right)\,\left(\mbox{{}1{}}-
 \mbox{{}0.5{}}\right)\,\left(\mbox{{}1{}}-\mbox{{}2{}}\right)}}\nonumber\\+{{
 \mbox{{}2{}}\,\left(x-\mbox{{}0{}}\right)\,\left(x-\mbox{{}0.5{}}
 \right)\,\left(x-\mbox{{}1{}}\right)}\over{\left(\mbox{{}2{}}-
 \mbox{{}0{}}\right)\,\left(\mbox{{}2{}}-\mbox{{}0.5{}}\right)\,
 \left(\mbox{{}2{}}-\mbox{{}1{}}\right)}}\nonumber\\={{\left(8\,x^3-24\,x^2+16\,
 x\right)\,y-16\,x^3+42\,x^2-17\,x}\over{3}\nonumber}
\end{eqnarray}

$$p\left(x\right)={{\left(8\,x^3-24\,x^2+16\,x\right)\,y-16\,x^3+42\,
 x^2-17\,x}\over{3}}$$

luego se encuentra que el coeficiente de $x^3$ es $(8y-16)/3$
que es quivalente a $6$ como lo indica el ejercicio:

$$\frac{8y-16}{3}=6$$
$$8y-16=18$$
$$8y=18+16=34$$
$$y=\frac{34}{8}=\frac{17}{4}=4.25$$

\section{ejercicio 10}
Sea $f\left(x\right)=\sqrt{x-x^2}$ y $P_2\left(x\right)$ el polinomio
de interpolacion con $x_0=0$, $x_1$ y $x_2=1$. Encuentre el valor de
$x_1$ en $(0,\;1)$ para que
$f\left(0.5\right)-P_2\left(0.5\right)=-0.25$.

Del enunciado se tiene
$$f\left(0.5\right)-P_2\left(0.5\right)=-0.25$$
evaluando la funcion $f$ en el valor $0.5$ se tiene $f(0.5)=0.5$, entonces
$$f\left(0.5\right)+0.25=P_2\left(0.5\right)$$
$$0.5+0.25=P_2\left(0.5\right)$$
$$P_2\left(0.5\right)=0.75$$
ahora si $x_1=m$ y $f(x_1)=\sqrt{x_1-x_1^2}=n$, haciendo la interpolación

\begin{eqnarray}
  {{\mbox{{}0{}}\,\left(x-\mbox{{}1{}}\right)\,\left(x-m\right)
 }\over{\left(\mbox{{}0{}}-\mbox{{}1{}}\right)\,\left(\mbox{{}0{}}-m
 \right)}}+{{\mbox{{}0{}}\,\left(x-\mbox{{}0{}}\right)\,\left(x-m
 \right)}\over{\left(\mbox{{}1{}}-\mbox{{}0{}}\right)\,\left(
 \mbox{{}1{}}-m\right)}}\nonumber\\+{{n\,\left(x-\mbox{{}0{}}\right)\,\left(x-
 \mbox{{}1{}}\right)}\over{\left(m-\mbox{{}0{}}\right)\,\left(m-
 \mbox{{}1{}}\right)}}={{n\,x^2-n\,x}\over{m^2-m}\nonumber}
\end{eqnarray}

$$p\left(x\right)={{n\,x^2-n\,x}\over{m^2-m}}$$

deshaciendo las sustituciones
$$p\left(x\right)={{x^2\,\sqrt{{\it x_1}-{\it x_1}^2}-x\,\sqrt{{\it x_1}-{\it x_1}^2}}\over{{\it x_1}^2-{\it x_1}}}$$
sabiendo que $P_2\left(0.5\right)=0.75$ se puede
$${{x^2\,\sqrt{{\it x_1}-{\it x_1}^2}-x\,\sqrt{{\it x_1}-{\it x_1}^2}}\over{{\it x_1}^2-{\it x_1}}}=0.75$$
de la misma fuente $P_2\left(0.5\right)=0.75$ se ve también que
$x=0.5$, sustituyendo este último y luego simplificando el resultante
$$-{{\sqrt{{\it x_1}-{\it x_1}^2}}\over{4\,{\it x_1}^2-4\,{\it x_1}}}=
 {{3}\over{4}}$$

aplicando el método de newton para encontrar $x_1$ se hace
$$f\left(x_1\right)=-{{\sqrt{{\it x_1}-{\it x_1}^2}}\over{4\,{\it x_1}^2-4\,{\it x_1}}}-{{3}\over{4}}=0$$
\begin{verbatim}
f(x):=-sqrt(x-x^2)/(4*x^2-4*x)-3/4;
newton(0.872,7,20);
\end{verbatim}
$$\pmatrix{N&{\it X\_n}&{\it X\_n}+1&{\it error}\cr 1&0.872&
 0.87268093&6.809250663 \times 10^{-4}\cr 2&0.87268093&0.872678&
 2.9287616645 \times 10^{-6}\cr 3&0.872678&0.872678&
 5.4665716398 \times 10^{-11}\cr }$$

por tanto, $x_1$ vale $0.872678$

\section{ejercicio 12}
Utilice el polinomio de interpolación de Lagrange de grado tres o
menos, aproximando a cuatro digitos, para calcular cos(0.750). Utilice
los siguientes valores.
$$\cos(0.698)=0.7661\quad \cos(0.733)=0.7432 \quad  \cos(0.768)=0.7193 \quad   \cos(0.803)=0.6946$$

se va ha utilizar
$$\pmatrix{0.733&0.768&0.803\cr 0.7432&0.7193&0.6946\cr }$$
\begin{verbatim}
  a:matrix([0.733,0.768,0.803],[0.7432,0.7193,0.6946]);
  polagrange(a);
\end{verbatim}

\begin{eqnarray}
  {{\mbox{{}0.7432{}}\,\left(x-\mbox{{}0.768{}}\right)\,\left(x-
 \mbox{{}0.803{}}\right)}\over{\left(\mbox{{}0.733{}}-
 \mbox{{}0.768{}}\right)\,\left(\mbox{{}0.733{}}-\mbox{{}0.803{}}
 \right)}}+{{\mbox{{}0.7193{}}\,\left(x-\mbox{{}0.733{}}\right)\,
 \left(x-\mbox{{}0.803{}}\right)}\over{\left(\mbox{{}0.768{}}-
 \mbox{{}0.733{}}\right)\,\left(\mbox{{}0.768{}}-\mbox{{}0.803{}}
 \right)}}\nonumber\\+{{\mbox{{}0.6946{}}\,\left(x-\mbox{{}0.733{}}\right)\,
 \left(x-\mbox{{}0.768{}}\right)}\over{\left(\mbox{{}0.803{}}-
 \mbox{{}0.733{}}\right)\,\left(\mbox{{}0.803{}}-\mbox{{}0.768{}}
 \right)}}=-{{4000000\,x^2+2361000\,x-12983969}\over{12250000}\nonumber}
\end{eqnarray}

$$p\left(x\right)=-{{4000000\,x^2+2361000\,x-12983969}\over{12250000
 }}$$

$$p\left(0.75\right)=0.73169135$$

\section{ejercicio 13}
Construir los polinomios de interpolación de Lagrange para las siguientes
funciones.



\subsection{inciso b)}
$f\left(x\right)=\sin(ln\;x) \quad x_0=2.0,\;x_1=2.4,\;x_2=2.6$
\begin{verbatim}
  a:matrix([2,2.4,2.6]);
  ln(x):=log(x)/log(%e);
  f(x):=sin(ln(x));
  a:addrow(a,map(f,a));
  polagrange(a);
\end{verbatim}
entonces tenemos los datos
$$\pmatrix{2&2.4&2.6\cr 0.63896128&0.76784388&0.81660905\cr }$$
haciendo la interpolación
\begin{eqnarray}
  {{\mbox{{}0.63896128{}}\,\left(x-\mbox{{}2.4{}}\right)\,\left(x-
 \mbox{{}2.6{}}\right)}\over{\left(\mbox{{}2{}}-\mbox{{}2.4{}}\right)
 \,\left(\mbox{{}2{}}-\mbox{{}2.6{}}\right)}}+{{\mbox{{}0.76784388{}}
 \,\left(x-\mbox{{}2{}}\right)\,\left(x-\mbox{{}2.6{}}\right)}\over{
 \left(\mbox{{}2.4{}}-\mbox{{}2{}}\right)\,\left(\mbox{{}2.4{}}-
 \mbox{{}2.6{}}\right)}}\nonumber\\+{{\mbox{{}0.81660905{}}\,\left(x-
 \mbox{{}2{}}\right)\,\left(x-\mbox{{}2.4{}}\right)}\over{\left(
 \mbox{{}2.6{}}-\mbox{{}2{}}\right)\,\left(\mbox{{}2.6{}}-
 \mbox{{}2.4{}}\right)}}=0.13063441\,x^2-0.89699789\,x+0.63249687 \nonumber
\end{eqnarray}

$$p\left(x\right)=0.13063441\,x^2-0.89699789\,x+0.63249687$$

\subsection{inciso d)}
$f\left(x\right)=\cos(x)+\sin(x),\quad x_0=0,\;x_1=0.25,\;x_2=0.5,\;x_3=1.0$. Los datos a utilizar son
$$\pmatrix{0&0.25&0.5&1\cr 1&1.21631638&1.357008100&
 1.38177329\cr }$$


\begin{verbatim}
  a:matrix([0,0.25,0.5,1]);
  f(x):=cos(x)+sin(x);
  a:addrow(a,map(f,a));
  polagrange(a);
\end{verbatim}

\begin{eqnarray}
  {{\mbox{{}1{}}\,\left(x-\mbox{{}0.25{}}\right)\,\left(x-
 \mbox{{}0.5{}}\right)\,\left(x-\mbox{{}1{}}\right)}\over{\left(
 \mbox{{}0{}}-\mbox{{}0.25{}}\right)\,\left(\mbox{{}0{}}-
 \mbox{{}0.5{}}\right)\,\left(\mbox{{}0{}}-\mbox{{}1{}}\right)}}\nonumber\\+{{
 \mbox{{}1.2163{}}\,\left(x-\mbox{{}0{}}\right)\,\left(x-
 \mbox{{}0.5{}}\right)\,\left(x-\mbox{{}1{}}\right)}\over{\left(
 \mbox{{}0.25{}}-\mbox{{}0{}}\right)\,\left(\mbox{{}0.25{}}-
 \mbox{{}0.5{}}\right)\,\left(\mbox{{}0.25{}}-\mbox{{}1{}}\right)}}\nonumber\\+
 {{\mbox{{}1.357{}}\,\left(x-\mbox{{}0{}}\right)\,\left(x-
 \mbox{{}0.25{}}\right)\,\left(x-\mbox{{}1{}}\right)}\over{\left(
 \mbox{{}0.5{}}-\mbox{{}0{}}\right)\,\left(\mbox{{}0.5{}}-
 \mbox{{}0.25{}}\right)\,\left(\mbox{{}0.5{}}-\mbox{{}1{}}\right)}}\nonumber\\+
 {{\mbox{{}1.3818{}}\,\left(x-\mbox{{}0{}}\right)\,\left(x-
 \mbox{{}0.25{}}\right)\,\left(x-\mbox{{}0.5{}}\right)}\over{\left(
 \mbox{{}1{}}-\mbox{{}0{}}\right)\,\left(\mbox{{}1{}}-\mbox{{}0.25{}}
 \right)\,\left(\mbox{{}1{}}-\mbox{{}0.5{}}\right)}}\\=-{{1192\,x^3+
 8178\,x^2-15097\,x-15000}\over{15000}}\nonumber
\end{eqnarray}

$$p\left(x\right)=-{{1192\,x^3+8178\,x^2-15097\,x-15000}\over{15000}}$$

\section{ejercicio 18}

\subsection{inciso a)}
Un censo de la poblacion de Estados Unidos da los resultados en la siguiente
tabla, en miles, desde 1950 a 200.
$$\pmatrix{
  Anio        & 1950   & 1960   & 1970   & 1980   &1990    & 2000   \cr 
  Poblacion  & 151326 & 179323 & 203302 & 226542 & 249633 & 281422 \cr 
}$$
Use interpolacion de Lagrange para aproximar la poblacion en los años
1940,1975 y 2020.

\subsubsection{Para año 1940}
datos a utilizar
$$\pmatrix{1950&1960\cr 151326&179323\cr }$$
haciendo los calculos
\begin{verbatim}
  a:matrix([1950,1960],[151326,179323])
  polagrange(a);
  p(1940);
\end{verbatim}
$${{\mbox{{}(151326){}}\,\left(x-\mbox{{}1960{}}\right)}\over{
 \mbox{{}1950{}}-\mbox{{}1960{}}}}+{{\mbox{{}(179323){}}\,\left(x-
 \mbox{{}1950{}}\right)}\over{\mbox{{}1960{}}-\mbox{{}1950{}}}}=0.1\,
 \left(27997\,x-53080890\right)$$
$$p\left(x\right)=0.1\,\left(27997\,x-53080890\right)$$
$$p\left(1940\right)=123329.0$$

\subsubsection{Para año 1975}
datos a utilizar
$$\pmatrix{1970&1980\cr 203302&226542\cr }$$
haciendo los calculos
\begin{verbatim}
  a:matrix([1970,1980],[203302,226542]);
  polagrange(a);
  p(1975);
\end{verbatim}
$${{\mbox{{}(203302){}}\,\left(x-\mbox{{}1980{}}\right)}\over{
 \mbox{{}1970{}}-\mbox{{}1980{}}}}+{{\mbox{{}(226542){}}\,\left(x-
 \mbox{{}1970{}}\right)}\over{\mbox{{}1980{}}-\mbox{{}1970{}}}}=2324
 \,x-4374978$$
$$p\left(x\right)=2324\,x-4374978$$
$$p\left(1975\right)=214922$$

\subsubsection{Para año 2020}
datos a utilizar
$$\pmatrix{1990&2000\cr 249633&281422\cr }$$
haciendo los calculos
\begin{verbatim}
  a:matrix([1990,2000],[249633,281422]);
  polagrange(a);
  p(2020);
\end{verbatim}

$${{\mbox{{}(249633){}}\,\left(x-\mbox{{}2000{}}\right)}\over{
 \mbox{{}1990{}}-\mbox{{}2000{}}}}+{{\mbox{{}(281422){}}\,\left(x-
 \mbox{{}1990{}}\right)}\over{\mbox{{}2000{}}-\mbox{{}1990{}}}}=0.1\,
 \left(31789\,x-60763780\right)$$
$$p\left(x\right)=0.1\,\left(31789\,x-60763780\right)$$
$$p\left(2020\right)=345000.0$$

\section{ejercicio 19}
Se sospecha que las altas cantidades de tanino en las hojas del roble
maduras inhiben el crecimiento, en invierno, de polilla, larvas que
dañan mucho estos árboles en ciertos años. La siguiente tabla muestra
el peso promedio de dos muestras de larvas en los primeros 28 días
después del nacimiento. La primera muestra se crió en las hojas de
roble jóvenes, mientras que la segunda muestra fue criados en hojas
maduras del mismo árbol.

\begin{itemize}
\item[a)] Utilizar la interpolación de Lagrange para aproximar la curva
  del peso promedio para cada muestra
\item[b)] Encontrar un peso promedio máximo aproximado para cada muestra
  por la determinación del máximo de la polinomio de interpolación.
\end{itemize}

\begin{table}[h]
  \centering
  \begin{tabular}[h]{l|ccccccc}
    Dia              & 0    & 6      & 10     & 13    & 17    & 20    & 28\\
    \hline
    Muestra 1 (mg)   & 6.67 & 17.33  & 42.67  & 37.33 & 30.10 & 29.31 & 28.74 \\
    Muestra 2 (mg)   & 6.67 & 16.11  & 18.89  & 15.00 & 10.56 & 9.44  & 8.89 
  \end{tabular}
  \caption{Peso promedio de dos muestras de larvas.}
  \label{tab:1}
\end{table}

\subsection{inciso a)}
Se escogen los puntos mas apropiados; aquellos en los que los pesos de la 
\textbf{muestra uno} alcanzan sus puntos mas altos. Utilizando los datos:
$$\pmatrix{6&10&13\cr 17.33&42.67&37.33\cr }$$

\begin{verbatim}
 a:matrix([6,10,13],[17.33,42.67,37.33]);
 polagrange(a);
\end{verbatim}

\begin{eqnarray}
{{\mbox{{}42.67{}}\,\left(x-\mbox{{}13{}}\right)\,\left(x-
 \mbox{{}6{}}\right)}\over{\left(\mbox{{}10{}}-\mbox{{}13{}}\right)\,
 \left(\mbox{{}10{}}-\mbox{{}6{}}\right)}}+{{\mbox{{}37.33{}}\,\left(
 x-\mbox{{}10{}}\right)\,\left(x-\mbox{{}6{}}\right)}\over{\left(
 \mbox{{}13{}}-\mbox{{}10{}}\right)\,\left(\mbox{{}13{}}-\mbox{{}6{}}
 \right)}}\nonumber\\+{{\mbox{{}17.33{}}\,\left(x-\mbox{{}10{}}\right)\,\left(x-
 \mbox{{}13{}}\right)}\over{\left(\mbox{{}6{}}-\mbox{{}10{}}\right)\,
 \left(\mbox{{}6{}}-\mbox{{}13{}}\right)}}=-
 7.14285 \times 10^{-4}\,\left(1623\,x^2-34837\,x+126332
 \right)\nonumber
 \end{eqnarray}

$$p\left(x\right)=-7.14285714 \times 10^{-4}\,\left(1623\,x^2
 -34837\,x+126332\right)$$

Se escogen los puntos mas apropiados; aquellos en los que los pesos de la 
\textbf{muestra dos} alcanzan sus puntos mas altos. Utilizando los datos:
$$\pmatrix{6&10&13\cr 16.11&18.89&15\cr }$$

\begin{verbatim}
  a:matrix([6,10,13],[16.11,18.89,15]);
  polagrange(a);
\end{verbatim}

\begin{eqnarray}
{{\mbox{{}18.89{}}\,\left(x-\mbox{{}13{}}\right)\,\left(x-
 \mbox{{}6{}}\right)}\over{\left(\mbox{{}10{}}-\mbox{{}13{}}\right)\,
 \left(\mbox{{}10{}}-\mbox{{}6{}}\right)}}+{{\mbox{{}15{}}\,\left(x-
 \mbox{{}10{}}\right)\,\left(x-\mbox{{}6{}}\right)}\over{\left(
 \mbox{{}13{}}-\mbox{{}10{}}\right)\,\left(\mbox{{}13{}}-\mbox{{}6{}}
 \right)}}\nonumber\\+{{\mbox{{}16.11{}}\,\left(x-\mbox{{}10{}}\right)\,\left(x-
 \mbox{{}13{}}\right)}\over{\left(\mbox{{}6{}}-\mbox{{}10{}}\right)\,
 \left(\mbox{{}6{}}-\mbox{{}13{}}\right)}}=-
 2.38095 \times 10^{-4}\,\left(1195\,x^2-22039\,x+21552
 \right)\nonumber
\end{eqnarray}

$$p\left(x\right)=-2.380952 \times 10^{-4}\,\left(1195\,x^2-
 22039\,x+21552\right)$$

\subsection{inciso b)}
Para la muestra uno
$${{d}\over{d\,x}}\,\left(-7.1428 \times 10^{-4}\,\left(
 1623\,x^2-34837\,x+126332\right)\right)=-
 7.142857 \times 10^{-4}\,\left(3246\,x-34837\right)$$

$$p\,'\left(x\right)=- 7.142857 \times 10^{-4}\,\left(3246\,x-34837\right)=0$$
$$x = 10.73228589032656$$
$$p\left(10.73228589032656\right)=43.29165841475221$$

Para la muestra dos
$${{d}\over{d\,x}}\,\left(-2.3809 \times 10^{-4}\,\left(
 1195\,x^2-22039\,x+21552\right)\right)=-
 2.38095 \times 10^{-4}\,\left(2390\,x-22039\right)$$

$$p\,'\left(x\right)=-2.38095 \times 10^{-4}\,\left(2390\,x-22039\right)=0$$
$$x = 9.221338912133891$$
$$p\left(19.06251051006176\right)=19.06251051006176$$

%%% Local Variables:
%%% TeX-master: "tarea1"
%%% End:
