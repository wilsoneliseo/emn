% pagina 90, seccion 2.5
\section{ejercicio 4}
Sea \(g(x)=1+(sin\,x)^2\) y \(p_0=1\). Use el método de steffensen
para encontrar \(p_{n+2}^{(1)}\) y \(p_{n+2}^{(2)}\)

\begin{verbatim}
g(x):=1+(sin(x))^2;
steffensen(1, 4, 20);
\end{verbatim}


$$\pmatrix{k & p_0^{(k)} & p_1^{(k)} & p_2^{(k)} & p_{n+2} & error\cr 1&1&1.708073&1.981273&2.152905&
 1.152905\cr 2&2.152905&1.697735&1.983973&1.873464&0.27944\cr }$$

por lo tanto \(p_{n+2}^{(1)}=2.152905\) y \(p_{n+2}^{(2)}=1.873464\)

\section{ejercicio 8}
Use el método de steffensen para encontrar, con una tolerancia menor a
$10^{-4}$, la raiz de $x-2^{-x}=0$ en el intervalo $[0,\,1]$

\begin{verbatim}
g(x):=2^(-x);
steffensen(0,4,20);
\end{verbatim}

$$\pmatrix{k & p_0^{(k)} & p_1^{(k)} & p_2^{(k)} & p_{n+2} & error\cr 1&0&1&0.5&0.66667&0.66667\cr 2&
 0.66667&0.62996&0.64619&0.64122&0.02545\cr 3&0.64122&0.64117&0.64119
 &0.64119&3.0447173 \times 10^{-5}\cr }$$

por lo tanto la solución aproximada es 0.64119

\section{ejercicio 9}
Use el método de steffensen con $p_0=2$ y calcule una aproximación de
$\sqrt{3}$ con una tolerancia $10^{-4}$.

\begin{verbatim}
g(x):=0.5*(x+3/x);
steffensen(2, 4, 20);
\end{verbatim}

$$\pmatrix{k & p_0^{(k)} & p_1^{(k)} & p_2^{(k)} & p_{n+2} & error\cr 1&2&1.75&1.732143&1.730769&0.26923
 \cr 2&1.730769&1.732051&1.732051&1.732051&0.0012816\cr 3&1.732051&
 1.732051&1.732051&1.732051&1.756042 \times 10^{-10}\cr }$$

por la aproximación a $\sqrt{3}$ es $1.732051$

\section{ejercicio 12}

\subsection{inciso b)}

Use el método de steffensen para aproximar la solución de $x^3-2x-5=0$
con una toleracia menor a $10^{-5}$. Use $g(x)=\sqrt[3]{2x+5}$.

\begin{verbatim}
g(x):=(2*x+5)^(1/3);
steffensen(1, 5, 20);
\end{verbatim}

$$\pmatrix{k & p_0^{(k)} & p_1^{(k)} & p_2^{(k)} & p_{n+2} & error\cr 1&1&1.9129312&2.0665808&2.0976736&
 1.0976736\cr 2&2.0976736&2.0950258&2.0946236&2.0945515&0.00312207
 \cr 3&2.0945515&2.0945515&2.0945515&2.0945515&
 1.92422669 \times 10^{-8}\cr }$$

por lo tanto la aproximación a la solucion es $2.0945515$

\subsection{inciso c)}

Use el método de steffensen para aproximar la solución de $3\,x^2-e^x=0$
con una toleracia menor a $10^{-5}$. Use $g(x)=\sqrt{\frac{e^x}{3}}$.

\begin{verbatim}
g(x):=sqrt((%e^x)/(3));
steffensen(0, 5, 20);
\end{verbatim}

$$\pmatrix{k & p_0^{(k)} & p_1^{(k)} & p_2^{(k)} & p_{n+2} & error\cr 1&0&0.57735&0.770565&0.86775&
 0.86775\cr 2&0.86775&0.890982&0.901392&0.909844&0.0420938\cr 3&
 0.909844&0.909933&0.909974&0.910008&1.64019988 \times 10^{-4}\cr 4&
 0.910008&0.910008&0.910008&0.910008&2.55462373 \times 10^{-9}\cr }$$

por lo tanto la aproximación a la solucion es $0.910008$

%%% Local Variables:
%%% TeX-master: "tarea1"
%%% End:
