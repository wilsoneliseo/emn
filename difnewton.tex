%EJERCICIOS SECCION 3.3
\section{ejercicio 1}

Por \textit{diferencias divididas de Newton} obtenga polinomios de
interpolación grados uno, dos y tres

\subsection{inciso b)}

$f(0.9)$ si $f(0.6) = -0.17694460$ $f(0.7) = 0.01375227$ $f(0.8) =
0.22363362$ $f(1.0) =0.65809197$

\subsubsection{Grado uno}
usando los datos
$$\pmatrix{0.8&1.0\cr 0.22363362&0.65809197\cr }$$
\begin{verbatim}
a:matrix([0.8,1.0],[0.22363362,0.65809197]);
difnewton(a);
\end{verbatim}

$$\pmatrix{\mbox{{}X{}}&\mbox{{}F(x){}}&\mbox{{}Columna 1{}}\cr 0.8&
 0.22363362&0\cr 1.0&0.65809197&2.17229175\cr }$$

Polinomio de diferencias progresivas
$$pProg(x)=2.17229175\,\left(x-0.8\right)+0.22363362$$

Polinomio de diferencias regresivas
$$pReg(x)=2.17229175\,\left(x-1.0\right)+0.65809197$$

Interpolación
$${\it pProg}\left(0.9\right)=0.440862795$$

\subsubsection{Grado dos}
datos a utilizar
$$\pmatrix{0.7&0.8&1.0\cr 0.01375227&0.22363362&0.65809197\cr }$$
\begin{verbatim}
a:matrix([0.7,0.8,1.0],[0.01375227,0.22363362,0.65809197]);
difnewton(a);
\end{verbatim}
$$\pmatrix{\mbox{{}X{}}&\mbox{{}F(x){}}&\mbox{{}Columna 1{}}&
 \mbox{{}Columna 2{}}\cr 0.7&0.01375227&0&0\cr 0.8&0.22363362&
 2.098813499999998&0\cr 1.0&0.65809197&2.17229175&0.24492750000001
 \cr }$$

Polinomio de diferencias progresivas
$$pProg(x)=0.2449275\,\left(x-0.8\right)\,\left(x-0.7\right)+
 2.0988134\,\left(x-0.7\right)+0.01375$$

Polinomio de diferencias regresivas
$$pReg(x)=0.24492750000001\,\left(x-1.0\right)\,\left(x-0.8\right)+2.17229175
 \,\left(x-1.0\right)+0.65809197$$

Interpolación
$${\it pProg}\left(0.9\right)=0.43841352$$

\subsubsection{Grado tres}
datos a utilizar
$$\pmatrix{0.6&0.7&0.8&1.0\cr -0.1769446&0.01375227&0.22363362&
 0.65809197\cr }$$
\begin{verbatim}
a:matrix([0.6,0.7,0.8,1.0],[-0.1769446,0.01375227,0.22363362,0.65809197]);
difnewton(a);
\end{verbatim}
$$\pmatrix{\mbox{{}X{}}&\mbox{{}F(x){}}&\mbox{{}Columna 1{}}&
 \mbox{{}Columna 2{}}&\mbox{{}Columna 3{}}\cr 0.6&-0.1769446&0&0&0
 \cr 0.7&0.01375227&1.906968701&0&0\cr 0.8&0.22363362&
 2.098813498&0.9592239&0\cr 1.0&0.65809197&2.17229175&
 0.244927501&-1.785741249949\cr }$$

Polinomio de diferencias progresivas
$$pProg(x)=-1.7857\,\left(x-0.8\right)\,\left(x-0.7\right)\,\left(x-
 0.6\right)+0.9592\,\left(x-0.7\right)\,\left(x-0.6\right)+
 1.9069\,\left(x-0.6\right)-0.17694$$

Polinomio de diferencias regresivas
$$pReg(x)=-1.7857\,\left(x-1.0\right)\,\left(x-0.8\right)\,\left(x-
 0.7\right)+0.2449\,\left(x-1.0\right)\,\left(x-0.8\right)+
 2.1722\,\left(x-1.0\right)+0.6580$$

Interpolación
$${\it pProg}\left(0.9\right)=0.4419850025$$

\section{ejercicio 3}

Por \textit{diferencias divididas de Newton} obtenga polinomios de
interpolación grados uno, dos y tres

\subsection{inciso b)}
$f(0.25) \;si\; f (0.1) = -0.62049958, f (0.2) = -0.28398668, f (0.3) = 0.00660095, f (0.4) = 0.24842440$

\subsubsection{Grado uno}
datos a utilizar
$$\pmatrix{0.2&0.3\cr -0.28398668&0.00660095\cr }$$
%--
\begin{verbatim}
a:matrix([0.2,0.3],[-0.28398668,0.00660095]);
difnewton(a);
\end{verbatim}
$$\pmatrix{\mbox{{}X{}}&\mbox{{}F(x){}}&\mbox{{}Columna 1{}}\cr 0.2&-
 0.28398668&0\cr 0.3&0.00660095&2.905876300000001\cr }$$

Polinomio de diferencias progresivas
$$pProg(x)=2.905876300000001\,\left(x-0.2\right)-0.28398668$$

Polinomio de diferencias regresivas
$$pReg(x)=2.905876300000001\,\left(x-0.3\right)+0.00660095$$

Interpolación
$${\it pProg}\left(0.25\right)=-0.138692865$$


\subsubsection{Grado dos}
datos a utilizar
$$\pmatrix{0.2&0.3&0.4\cr -0.28398668&0.00660095&0.2484244\cr }$$
\begin{verbatim}
a:matrix([0.2,0.3,0.4],[-0.28398668,0.00660095,0.2484244]);
difnewton(a);
\end{verbatim}
$$\pmatrix{\mbox{{}X{}}&\mbox{{}F(x){}}&\mbox{{}Columna 1{}}&
 \mbox{{}Columna 2{}}\cr 0.2&-0.28398668&0&0\cr 0.3&0.00660095&
 2.905876300000001&0\cr 0.4&0.2484244&2.418234499999999&-
 2.438209000000007\cr }$$

Polinomio de diferencias progresivas
$$pProg(x)=-2.438209\,\left(x-0.3\right)\,\left(x-0.2\right)+
 2.9058763\,\left(x-0.2\right)-0.28398$$

Polinomio de diferencias regresivas
$$pReg(x)=-2.438209\,\left(x-0.4\right)\,\left(x-0.3\right)+
 2.4182345\,\left(x-0.4\right)+0.2484$$

Interpolación
$${\it pProg}\left(0.25\right)=-0.1325973425$$

\subsubsection{Grado tres}
datos a utilizar
$$\pmatrix{0.1&0.2&0.3&0.4\cr -0.62049958&-0.28398668&0.00660095&
 0.2484244\cr }$$
\begin{verbatim}
a:matrix([0.1,0.2,0.3,0.4],[-0.62049958,-0.28398668,0.00660095,0.2484244]);
difnewton(a);
\end{verbatim}
$$\pmatrix{\mbox{{}X{}}&\mbox{{}F(x){}}&\mbox{{}Columna 1{}}&
 \mbox{{}Columna 2{}}&\mbox{{}Columna 3{}}\cr 0.1&-0.6204&0&0&0
 \cr 0.2&-0.2839&3.365129&0&0\cr 0.3&0.00660095&
 2.9058763&-2.2962&0\cr 0.4&0.24842&2.4182345&-
 2.438209&-0.473152\cr }$$


Polinomio de diferencias progresivas
$$pProg(x)=-0.473151\,\left(x-0.3\right)\,\left(x-0.2\right)\,\left(x-
 0.1\right)-2.296263\,\left(x-0.2\right)\,\left(x-0.1\right)+
 3.365129\,\left(x-0.1\right)-0.62049$$

Polinomio de diferencias regresivas
$$pReg(x)=-0.473152\,\left(x-0.4\right)\,\left(x-0.3\right)\,\left(x-
 0.2\right)-2.438209\,\left(x-0.4\right)\,\left(x-0.3\right)+
 2.4182345\,\left(x-0.4\right)+0.248424$$

Interpolación
$${\it pProg}\left(0.25\right)=-0.132774774375$$


\section{ejercicio 5}
Por \textit{diferencias divididas de Newton} obtenga polinomios de
interpolación grados uno, dos y tres

\subsection{inciso b)}

$f(0.25)$ si $f(0.1) = -0.62049958$, $f(0.2) = -0.28398668$,  $f(0.3) =
0.00660095$, $f(0.4) =0.24842440$

\subsubsection{Grado uno}
usando los datos
$$\pmatrix{0.2&0.3\cr -0.28398668&0.00660095\cr }$$
\begin{verbatim}
a:matrix([0.2,0.3],[-0.28398668,0.00660095]);
difnewton(a);
\end{verbatim}
$$\pmatrix{\mbox{{}X{}}&\mbox{{}F(x){}}&\mbox{{}Columna 1{}}\cr 0.2&-
 0.28398668&0\cr 0.3&0.00660095&2.905876300000001\cr }$$

Polinomio de diferencias progresivas
$$pProg(x)=2.905876300000001\,\left(x-0.2\right)-0.28398668$$

Polinomio de diferencias regresivas
$$pReg(x)=2.905876300000001\,\left(x-0.3\right)+0.00660095$$

Interpolación
$${\it pReg}\left(0.25\right)=-0.138692865$$


\subsubsection{Grado dos}
datos a utilizar
$$\pmatrix{0.2&0.3&0.4\cr -0.28398668&0.00660095&0.2484244\cr }$$
\begin{verbatim}
a:matrix([0.2,0.3,0.4],[-0.28398668,0.00660095,0.2484244]);
difnewton(a);
\end{verbatim}
$$\pmatrix{\mbox{{}X{}}&\mbox{{}F(x){}}&\mbox{{}Columna 1{}}&
 \mbox{{}Columna 2{}}\cr 0.2&-0.28398668&0&0\cr 0.3&0.00660095&
 2.905876300000001&0\cr 0.4&0.2484244&2.418234499999999&-
 2.438209000000007\cr }$$

Polinomio de diferencias progresivas
$$pProg(x)=-2.43820900000001\,\left(x-0.3\right)\,\left(x-0.2\right)+
 2.905876300000001\,\left(x-0.2\right)-0.28398668$$

Polinomio de diferencias regresivas
$$pReg(x)=-2.43820900000001\,\left(x-0.4\right)\,\left(x-0.3\right)+
 2.418234499999999\,\left(x-0.4\right)+0.2484244$$

Interpolación
$${\it pReg}\left(0.25\right)=-0.1325973425$$

\subsubsection{Grado tres}
datos a utilizar
$$\pmatrix{0.1&0.2&0.3&0.4\cr -0.62049958&-0.28398668&0.00660095&
 0.2484244\cr }$$
\begin{verbatim}
a:matrix([0.1,0.2,0.3,0.4],[-0.62049958,-0.28398668,0.00660095,0.2484244]);
difnewton(a);
\end{verbatim}

$$\pmatrix{\mbox{{}X{}}&\mbox{{}F(x){}}&\mbox{{}Columna 1{}}&
 \mbox{{}Columna 2{}}&\mbox{{}Columna 3{}}\cr 0.1&-0.62049958&0&0&0
 \cr 0.2&-0.28398668&3.365129000000001&0&0\cr 0.3&0.00660095&
 2.905876300000001&-2.2962635&0\cr 0.4&0.2484244&2.418234499999999&-
 2.438209000000007&-0.47315166666669\cr }$$

Polinomio de diferencias progresivas
$$pProg(x)=-0.47315\,\left(x-0.3\right)\,\left(x-0.2\right)\,\left(x-
 0.1\right)-2.2962\,\left(x-0.2\right)\,\left(x-0.1\right)+
 3.365129\,\left(x-0.1\right)-0.62049$$

Polinomio de diferencias regresivas
$$pReg(x)=-0.473152\,\left(x-0.4\right)\,\left(x-0.3\right)\,\left(x-
 0.2\right)-2.438209\,\left(x-0.4\right)\,\left(x-0.3\right)+
 2.4182345\,\left(x-0.4\right)+0.24842$$

Interpolación
$${\it pReg}\left(0.25\right)=-0.132774774375$$

\section{ejercicio 7}

\subsection{inciso a)}
Use interpolación, de grado tres, por diferencias divididas de Newton,
para los siguientes puntos.


\begin{table}[H]
  \centering
  \begin{tabular}{rl}
    $x$ & $f(x)$ \\ \hline
    -0.1&5.3 \\
    0&2 \\ 
    0.2&3.19\\ 
    0.3&1 \\ \hline
  \end{tabular}
\end{table}

{\Large \noindent Solucion:}

\begin{verbatim}
a:matrix([-0.1,0,0.2,0.3],[5.3,2,3.19,1]);
difnewton(a);
\end{verbatim}

$$\pmatrix{\mbox{{}X{}}&\mbox{{}F(x){}}&\mbox{{}Columna 1{}}&
 \mbox{{}Columna 2{}}&\mbox{{}Columna 3{}}\cr -0.1&5.3&0&0&0\cr 0&2&-
 32.99&0&0\cr 0.2&3.19&5.9499&
 129.83&0\cr 0.3&1&-21.9&-92.833&-
 556.6666\cr }$$

Polinomio Diferencias progresivas
$$pProg(x)=-556.6666\,\left(x-0.2\right)\,x\,\left(x+0.1\right)+
 129.8333\,x\,\left(x+0.1\right)-33.0\,\left(x+0.1\right)+
 5.3$$

Polinomio Diferencias Regresivas
$$pReg(x)=-556.666\,\left(x-0.3\right)\,\left(x-0.2\right)\,x-
 92.833\,\left(x-0.3\right)\,\left(x-0.2\right)-21.9\,
 \left(x-0.3\right)+1$$

\subsection{inciso b)}
Agrege $f(0.35)=0.97260$ a los puntos y construya el polinomio de
interpolación de grado cuatro.

{\Large \noindent Solucion:}

\begin{verbatim}
a:matrix([-0.1,0,0.2,0.3,0.35],[5.3,2,3.19,1,0.97260]);
difnewton(a);
\end{verbatim}

$$\pmatrix{\mbox{{}X{}}&\mbox{{}F(x){}}&\mbox{{}Columna 1{}}&
 \mbox{{}Columna 2{}}&\mbox{{}Columna 3{}}&\mbox{{}Columna 4{}}\cr -
 0.1&5.3&0&0&0&0\cr 0&2&-32.999&0&0&0\cr 0.2&3.19&
 5.9499&129.8333&0&0\cr 0.3&1&-21.900&
 -92.8333&-556.6666&0\cr 0.35&0.9726&-0.548&
 142.346666&671.94285&2730.2433862\cr }$$


Polinomio Diferencias progresivas
$$pProg(x)=2730.24\,\left(x-0.3\right)\,\left(x-0.2\right)\,x\,
 \left(x+0.1\right)-556.66\,\left(x-0.2\right)\,x\,\left(x+
 0.1\right)$$
$$+129.83\,x\,\left(x+0.1\right)-33.0\,\left(x+
 0.1\right)+5.3$$


Polinomio Diferencias Regresivas
$$pReg(x)=2730.24\,\left(x-0.35\right)\,\left(x-0.3\right)\,\left(x
 -0.2\right)\,x+671.942\,\left(x-0.35\right)\,\left(x-0.3 
 \right)\,\left(x-0.2\right)$$
$$+142.346\,\left(x-0.35\right)\,
 \left(x-0.3\right)-0.548\,\left(x-0.35\right)+0.9726$$

\section{ejercicio 9}

\subsection{inciso a)}
Aproximar $f(0.05)$ usando los siguientes datos, con la formula de
diferencias progresivas:

\begin{table}[H]
  \centering
  \begin{tabular}{c|l|l|l|l|l}
    $x$&0.0&0.2&0.4&0.6&0.8\\\hline
    $f(x)$&1.00000&1.22140&1.49182&1.82212&2.22554
  \end{tabular}
\end{table}

\begin{verbatim}
a:matrix([0.0,0.2,0.4,0.6,0.8],[1.0,1.2214,1.49182,1.82212,2.22554]);
difnewton(a);
\end{verbatim}
$$\pmatrix{\mbox{{}X{}}&\mbox{{}F(x){}}&\mbox{{}Columna 1{}}&
 \mbox{{}Columna 2{}}&\mbox{{}Columna 3{}}&\mbox{{}Columna 4{}}\cr 
 0.0&1.0&0&0&0&0\cr 0.2&1.2214&1.107&0&0&0\cr 0.4&1.49182&
 1.352099&0.61275&0&0\cr 0.6&1.82212&1.6515&
 0.7485&0.22625&0\cr 0.8&2.22554&2.0171&0.914&
 0.27583&0.0619792\cr }$$

Formula de diferencias progresivas
$$pProg(x)=0.06197916\,\left(x-0.6\right)\,\left(x-0.4\right)\,\left(x-
 0.2\right)\,x+0.22625\,\left(x-0.4\right)\,\left(x-0.2
 \right)\,x$$
$$+0.61275\,\left(x-0.2\right)\,x+1.107\,x+1.0$$

Evaluación
$${\it pProg}\left(0.05\right)=1.051258798828125$$

\subsection{inciso b)}
Use la formula por diferencias regresivas de Newton para aproximar $f(0.65)$.\\

\noindent La formula de diferencias Regresivas es
$$pReg(x)=0.061\,\left(x-0.8\right)\,\left(x-0.6\right)\,\left(x-
 0.4\right)\,\left(x-0.2\right)+0.2758\,\left(x-0.8\right)
 \,\left(x-0.6\right)\,\left(x-0.4\right)$$
$$+0.914\,\left(x-0.8\right)\,
 \left(x-0.6\right)+2.0171\,\left(x-0.8\right)+2.22554$$

Evaluación
$${\it pReg}\left(0.65\right)=1.915550517578125$$

\section{ejercicio 13}%785
Los siguientes datos los da un polinomio $P(x)$ de grado desconocido.
\begin{table}[H]
  \centering
  \begin{tabular}{c|l|l|l}
    $x$&0&1&2\\\hline
    $P(x)$&2&-1&4
  \end{tabular}
\end{table}

Determine el coeficiente de $x^2$ en $P(x)$ si todos tercer-orden
de diferencias progresivas son 1.

\begin{verbatim}
a:matrix([0,1,2],[2,-1,4]);
difnewton(a);
\end{verbatim}

$$\pmatrix{\mbox{{}X{}}&\mbox{{}F(x){}}&\mbox{{}Columna 1{}}&
  \mbox{{}Columna 2{}}\cr 0&2&0&0\cr 1&-1&-3&0\cr 2&4&5&4\cr }$$

Polinomio de diferencias progresivas
$$pProg(x)=4\,\left(x-1\right)\,x-3\,x+2$$

Simplificando
$$pProg(x)=4\,x^2-7\,x+2$$

\section{ejercicio 16}
Para una funcion $f$, la formula de diferencia dividida de
interpolación da el polinomio

$$P_3(x)=1+4x+4x(x-0.25)+\frac{16}{3}x(x-0.25)(x-0.5),$$

con los nodos $x_0=0$, $x_1=0.25$, $x_2=0.5$, y $x_3=0.75$. Buscar $f(0.75).$

\begin{verbatim}
a:matrix([0,0.25,0.5,0.75],[1,m,n,p]);
difnewton(a);
\end{verbatim}
{\scriptsize
$$\pmatrix{\mbox{{}X{}}&\mbox{{}F(x){}}&\mbox{{}Columna 1{}}&
 \mbox{{}Columna 2{}}&\mbox{{}Columna 3{}}\cr 0&1&0&0&0\cr 0.25&m&4.0
 \,\left(m-1\right)&0&0\cr 0.5&n&4.0\,\left(n-m\right)&2.0\,\left(4.0
 \,\left(n-m\right)-4.0\,\left(m-1\right)\right)&0\cr 0.75&p&4.0\,
 \left(p-n\right)&2.0\,\left(4.0\,\left(p-n\right)-4.0\,\left(n-m
 \right)\right)&1.33\,\left(2.0\,\left(4.0\,\left(p-n
 \right)-4.0\,\left(n-m\right)\right)-2.0\,\left(4.0\,\left(n-m
 \right)-4.0\,\left(m-1\right)\right)\right)\cr }$$
}

ahora comparando 
$$P_3(x)=1+4x+4x(x-0.25)+\frac{16}{3}x(x-0.25)(x-0.5)$$
con la forma general
$$P_3(x)=a_0+a_1(x-x_0)+a_2(x-x_0)(x-x_1)+a_3(x-x_0)(x-x_1)(x-x_2)$$
se ve de esto que $a_0=1$, $a_1=1$, $a_2=1$ y $a_3=4/3$. Igualando
estos valores con los $a_0$, $a_1$, $a_2$, $a_3$ de la tabla
$$4.0\,\left(m-1\right)=1 \quad \quad \quad \quad m=1.25$$

$$2.0\,\left(4.0\,\left(n-1.25\right)-4.0\,\left(1.25-1\right)\right)=1 \quad \quad \quad \quad n=1.625$$

$$1.33333\,\left(2.0\,\left(4.0\,\left(p-1.625\right)-4.0\,
 \left(1.625-1.25\right)\right)-2.0\,\left(4.0\,\left(1.625-1.25\right)-4.0\,\left(
 1.25-1\right)\right)\right)=4/3 \quad \quad p=2.25$$

Luego $f(0.75)=2.25$.

\section{ejercicio 17}
Para una funcion $f$, las diferencias progresivas estan dados por

\begin{table}[H]
  \centering
  \begin{tabular}{clll}
    \hline
    $x_0=0.0$&$f[x_0]$& & \\ 
             &       & $f[x_0,x_1]$ & \\
    $x_1=0.4$&$f[x_1]$& & $f[x_0,x_1,x_2]=50/7$ \\
             &       & $f[x_1,x_2]=10$ & \\
    $x_2=0.7$&$f[x_2]=6$& & \\ \hline
  \end{tabular}
\end{table}
Complete la tabla.

\begin{verbatim}
a:matrix([0,0.4,0.7],[m,n,6]);
difnewton(a);
\end{verbatim}

$$\pmatrix{\mbox{{}X{}}&\mbox{{}F(x){}}&\mbox{{}Columna 1{}}&
 \mbox{{}Columna 2{}}\cr 0&m&0&0\cr 0.4&n&2.5\,\left(n-m\right)&0\cr 
 0.7&6&3.333333333333334\,\left(6-n\right)&1.4286\,\left(
 3.3333\,\left(6-n\right)-2.5\,\left(n-m\right)\right)\cr 
 }$$
De la ultima fila columna 1 y comparando con la tabla dada se tiene
$$3.333333333333334\,\left(6-n\right)=10$$
de esto se saca que $n=3$\\

De forma similar se ve que
$$1.42857\,\left(3.33333\,\left(6-n\right)-2.5\,
 \left(n-m\right)\right)=50/7$$
sutituyendo el valor de $n$ encontrado
$$1.42857\,\left(3.33333\,\left(6-3\right)-2.5\,
 \left(3-m\right)\right)=50/7$$
de aquí se saca que $m=1$

Por lo que la tabla es
$$\pmatrix{\mbox{{}X{}}&\mbox{{}F(x){}}&\mbox{{}Columna 1{}}&
 \mbox{{}Columna 2{}}\cr 0&1&0&0\cr 0.4&3&5.0&0\cr 0.7&6&10.0&
 7.142857142857146\cr }$$

%%% Local Variables:
%%% TeX-master: "tarea1"
%%% End:
